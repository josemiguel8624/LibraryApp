% Copie este bloque por cada caso de uso:
%-------------------------------------- COMIENZA descripción del caso de uso.

	\begin{UseCase}{CU07}{Prestar libros a domicilio}{
	Autorizar a los usuarios el poder retirar de la biblioteca los materiales con los que se cuentan en disposición de prestamo, mediante el llenado de un formulario donde se introducirá  el identificador del usuario que solicita este tramite y también el identificador del libro que se desea pedir en prestamo para que así el sistema autorize o no la solicitud.

	}
		\UCitem{Versión}{1.3}
		\UCitem{Autor}{Mora Guardado Daniel y Cortes Frias Diego Antonio.}
		\UCitem{Revisó}{Miguel Tejeda y Erubey Martinez.}
		\UCitem{Estado}{Aprobado}
		\UCitem{Actor}{Administrador}
		\UCitem{Propósito}{Autorizar el prestamo de un libro para que el usuario sea capaz de llevar el libro fuera biblioteca. }
		\UCitem{Entradas}{
						\begin{itemize}
							\item Identificador del usuario.
							\item Identificador del libro.
						\end{itemize}
					}
		\UCitem{Salidas}{Se redireccionará al usuario a una pantalla donde muestre los resultados de su peticion.}
		\UCitem{Precondiciones}{ 
			\begin{itemize}
				\item El administrador debe haber ingresado al sistema. 
				\item El libro debe estar disponible para prestamo y se debe de contar con su identificador.
				\item El usuario debe de estar dado de alta en el sistema y contar con un identificador.
			\end{itemize}
		}
		\UCitem{Postcondiciones}{Se actualiza en la base de datos  la lista de libros con los que el usuario cuenta en prestamo y cambiará el estado de libro a prestado. }

	\end{UseCase}
		%-------------------------------------- COMIENZA descripción Trayectoria Principal
	\begin{UCtrayectoria}{Principal}
		\UCpaso[\UCactor] Da click en  \IUbutton{Pretsamo de Libros} de la interfaz \IUref{UI01}{Pantalla de Inicio}  .
		\UCpaso[\UCactor] Llena el formulario de la interfaz \IUref{UI02}{Pantalla de prestamo a domicilio} \BRitem Se llena con el identificador del usuario y el identificador del libro que el usuario quiere como pretsamo  \Trayref{A}.
		\UCpaso[\UCactor] Da click en  \IUbutton{Realizar tramite} 
		\UCpaso[\UCsist] Se conecta a la BD \Trayref{B}.
		\UCpaso[\UCsist] Verifica la cuenta del usuario \Trayref{C} \Trayref{D} \Trayref{E} \Trayref{F}. .
		\UCpaso[\UCsist] Verifica la disponibilidad del libro \Trayref{G}.
		\UCpaso[\UCsist] Se muestra en \IUref{UI03}{Prestamo satisfacotrio} un mensaje con la fecha de devolucion del libro. 
		\UCpaso[\UCactor] Da click en  \IUbutton{Entendido} de  \IUref{UI03}{Prestamo satisfacotrio} y se redireccionara a  \IUref{UI01}{Pantalla de Inicio} .
	\end{UCtrayectoria}
			%-------------------------------------- COMIENZA descripción Trayectoria Alternativa.
		\begin{UCtrayectoriaA}{A}{No se llenan todos los campos del formulario de la \IUref{UI02}{Pantalla de prestamo a domicilio}  }
			\UCpaso[\UCsist] Vuelve a cargar la  \IUref{UI02}{Pantalla de prestamo a domicilio} con el mensaje \MSGref{E3}{Datos incompletos} en letrar rojas en la parte superior del formulario.
			\UCpaso[\UCsist] Regresa al paso 2 de la trayectoria principal.
		\end{UCtrayectoriaA}


			%-------------------------------------- COMIENZA descripción Trayectoria Alternativa.
		\begin{UCtrayectoriaA}{B}{No se pudó hacer la conexión a la base de datos.  }
			\UCpaso[\UCsist] Se muestra  \IUref{UI04}{Error}  con el mensaje \MSGref{E1}{Fallo de conexion}
			\UCpaso[\UCactor] Da click en  \IUbutton{Aceptar}.
			\UCpaso[\UCsist] Redirecciona a  \IUref{UI01}{Pantalla de Inicio}  
		\end{UCtrayectoriaA}

			%-------------------------------------- COMIENZA descripción Trayectoria Alternativa.
		\begin{UCtrayectoriaA}{C}{Se viola la \BRref{Rn4}{Usuario no registrado}  }
			\UCpaso[\UCsist] Se muestra  \IUref{UI04}{Error}   con el mensaje \MSGref{E4}{Usuario no registrado}
			\UCpaso[\UCactor] Da click en  \IUbutton{Aceptar}.
			\UCpaso[\UCsist] Redirecciona a  \IUref{UI01}{Pantalla de Inicio}  
		\end{UCtrayectoriaA}

			%-------------------------------------- COMIENZA descripción Trayectoria Alternativa.
		\begin{UCtrayectoriaA}{D}{Se viola la \BRref{Rn5}{Limite de prestamos}  }
			\UCpaso[\UCsist] Se muestra \IUref{UI04}{Error}   con el mensaje \MSGref{MSG7}{Limite de prestamos}
			\UCpaso[\UCactor] Da click en  \IUbutton{Aceptar}.
			\UCpaso[\UCsist] Redirecciona a  \IUref{UI01}{Pantalla de Inicio}   
		\end{UCtrayectoriaA}

			%-------------------------------------- COMIENZA descripción Trayectoria Alternativa.
		\begin{UCtrayectoriaA}{E}{Se viola la \BRref{Rn6}{Sancion de prestamos}  }
			\UCpaso[\UCsist] Se muestra  \IUref{UI04}{Error}  con el mensaje \MSGref{MSG2}{Sancion}
			\UCpaso[\UCactor] Da click en  \IUbutton{Aceptar}.
			\UCpaso[\UCsist] Redirecciona a  \IUref{UI01}{Pantalla de Inicio}   
		\end{UCtrayectoriaA}

			%-------------------------------------- COMIENZA descripción Trayectoria Alternativa.
		\begin{UCtrayectoriaA}{F}{Se viola la \BRref{Rn7}{Usuario bloqueado}  }
			\UCpaso[\UCsist] Se muestra  \IUref{UI04}{Error}   con el mensaje \MSGref{MSG4}{Usuario bloqueado}
			\UCpaso[\UCactor] Da click en  \IUbutton{Aceptar}.
			\UCpaso[\UCsist] Redirecciona a  \IUref{UI01}{Pantalla de Inicio}  
		\end{UCtrayectoriaA}

			%-------------------------------------- COMIENZA descripción Trayectoria Alternativa.

		\begin{UCtrayectoriaA}{G}{Se viola la \BRref{Rn8}{Estado del libro para prestamo}  }
			\UCpaso[\UCactor] Da click en  \IUbutton{Aceptar}.
			\UCpaso[\UCsist] Redirecciona a  \IUref{UI01}{Pantalla de Inicio} 

		\end{UCtrayectoriaA}










		
%-------------------------------------- TERMINA descripción del caso de uso.
