% Copie este bloque por cada caso de uso:
%-------------------------------------- COMIENZA descripción del caso de uso.

	\begin{UseCase}{CU02}{Generar cuentas de usuario}{
		En este caso de uso tiene como objetivo generar las cuentas de los nuevos usuarios al sistema.
	}
		\UCitem{Versión}{0.1}
		\UCitem{Actor}{Administrador}
		\UCitem{Propósito}{Generar nuevas cuentas de usuarios para que puedan acceder a los servicios ofrecidos por la biblioteca (consultar procesos de negocio). }
		\UCitem{Entradas}{
		\begin{itemize}
		\item Una instancia de la clase Usuario
		  \begin{itemize}
		   \item nombres[String]
		   \item apellidoPa[String]
		   \item apellidoMa[String]
		   \item email[String]
		   \item calle[String]
		   \item no[int]
		   \item cp[int]
		   \item curp[String]
		   \item telefono[String]
		   \item rol[int]
		   \item foto[file]
		  \end{itemize}
		\end{itemize}				
		}
		\UCitem{Salidas}{
		\begin{itemize}
		\item Se enviará un correo al usuario con sus datos de registro y su idUsuario
		\item Actualizará la capa de modelo
		\end{itemize}				
		}
		\UCitem{Precondiciones}{ 
			\begin{itemize}
				\item El administrador debe haber ingresado al sistema mediante el 			                 enlace. 
				\item El usuario no tuvo que haber sido registrado anteriormente
				\item Deberá existir el catálogo para usuarios
			\end{itemize}
		}
		\UCitem{Postcondiciones}{
		\begin{itemize}
		\item El usuario podrá autenticarse
		\item Acceso a las funciones correspondientes al usuario
		\end{itemize}				
		}
		\UCitem{Autor}{Burciaga Ornelas Rodrigo Andrés}
	\end{UseCase}
		%-------------------------------------- COMIENZA descripción Trayectoria Principal
	\begin{UCtrayectoria}{Principal}
	\UCpaso[\UCactor] Ingresa al sistema mediante el enlace a la aplicacion web
	 \UCpaso[\UCactor] Selecciona agregar usuario de la interfaz \IUref{UI01}{Gestión Usuario}.
	 \UCpaso[\UCsist]Muestra la interfaz \IUref{UI02}{Agregar Usuario}.
	  \UCpaso[\UCactor] Administrador llena el formulario de registro
	   \UCpaso[\UCsist] Verifica la  \BRref{RN33}{Datos del formulario completos}.[Trayectoria A]
	    \UCpaso[\UCsist] Verifica la  \BRref{RN34}{Datos congruentes en cada campo}.[Trayectoria B]
	    \UCpaso[\UCsist] Verifica la \BRref{RN35}{El usuario ha sido registrado anteriormente}.[Trayectoria C]
	  \UCpaso[\UCsist]Envía la información del input CURP a la appi de RENAPO.[Trayectoria D]
	  \UCpaso[\UCsist]Obtiene la información de la APPI de RENAPO. [Trayectoria E]
	  \UCpaso[\UCsist]El sistema valida la información ingresada por el usuario comparando con la enviada por la APPI. [Trayectoria F]
	    \UCpaso[\UCsist]Informa que el registro fue exitoso y envia un correo de confirmación al usuario.Se muestra en pantalla la \IUref{UI03}{Inserción Exitosa!}.
	\end{UCtrayectoria}
			%-------------------------------------- COMIENZA descripción Trayectoria Alternativa.
		\begin{UCtrayectoriaA}{A}{Campos sin llenar.}
			\UCpaso[\UCsist] El sistema envia el mensaje \MSGref{MSG01}{Debe Llenar todos los campos marcados con *}

		\end{UCtrayectoriaA}
		
		\begin{UCtrayectoriaA}{B}{Los datos ingresados no son correctos.}
			\UCpaso[\UCsist] El sistema envia el mensaje \MSGref{MSG02}{Los datos ingresados en el campo señalado no son congruentes}
		
		\end{UCtrayectoriaA}	
		
		
		\begin{UCtrayectoriaA}{C}{Usuario existente en la base.}
			\UCpaso[\UCsist] El sistema envia el mensaje  \MSGref{MSG03}{Usuario Registrado anteriormente}

		\end{UCtrayectoriaA}
		
		
		\begin{UCtrayectoriaA}{D}{La APPI no se encuentra disponible}
			\UCpaso[\UCsist] El sistema guarda la información en la base de datos.
			\UCpaso[\UCsist]Intenta enviar la información cada media hora hasta que este disponible.
		\end{UCtrayectoriaA}
		
		\begin{UCtrayectoriaA}{E}{No se pudieron obtener los datos}
			\UCpaso[\UCsist] El sistema envia el mensaje  \MSGref{MSG04}{Los datos no se pudieron obtener de la appi de Renapo}

		\end{UCtrayectoriaA}		
		
		
		
		\begin{UCtrayectoriaA}{E}{Los datos no coinciden}
			\UCpaso[\UCsist] El sistema envia el mensaje  \MSGref{MSG05}{Los datos ingresados no son correctos}

		\end{UCtrayectoriaA}
		

		\begin{UCtrayectoriaA}{F}{Los datos no coinciden}
			\UCpaso[\UCsist] El sistema envia el mensaje  \MSGref{MSG06}{Los datos ingresados de la CURP no son correctos}

		\end{UCtrayectoriaA}
		
		
		
		
		
%-------------------------------------- TERMINA descripción del caso de uso.