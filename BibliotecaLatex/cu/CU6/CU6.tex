% Copie este bloque por cada caso de uso:
%-------------------------------------- COMIENZA descripción del caso de uso.

	\begin{UseCase}{CU06}{Registrar Cambio de Estado de un Libro}{
		Proceso que se realiza después de la devolución  de un libro por parte del usuario a la biblioteca, en esta transición el libro adquiere un nuevo estado que lo establece las  características importantes descritos por la \BRref{RN30}.
	}
		\UCitem{Versión}{0.1}
		\UCitem{Actor}{Administrador bibliotecario}
		\UCitem{Propósito}{Definir el estado actual con el que cuenta un libro después de su entrega por parte del usuario para facilitar el control de inventario y futuras consultas.}
		\UCitem{Entradas}{El identificador del libro , la fecha -hora cuando se realiza el registro y los estados que se asocian al libro.}
		\UCitem{Salidas}{Se registra la fecha y hora del cambio de estado del libro , esto indica la disponibilidad y características durante ese cambio.}
		\UCitem{Precondiciones}{ 
			\begin{itemize}
			\item El Libro  debe existir en los registros.
 			\item  Se realizó el proceso de préstamo previamente \UCref(CU09)o se realizó en proceso de 						devolución  \UCref(CU07) para identificar el libro para realizar el cambio de estado.
			\end{itemize}
		}
		\UCitem{Postcondiciones}{El cambio de estado queda registrado con una fecha y hora, una disponibilidad y  0 o más condiciones físicas.}
		\UCitem{Autor}{BAUTISTA ROSALES MAURICIO}
	\end{UseCase}
		%-------------------------------------- COMIENZA descripción Trayectoria Principal
	\begin{UCtrayectoria}{Principal}

	\UCpaso[\UCactor] El administrador realiza el proceso de préstamo de libro o devolución  \UCref(CU7) , \UCref(CU09).
	\UCpaso[\UCsist]   identifica si se trata de un préstamo o devolución con base a la regla de negocio \BRref{RN31}.
 	\UCpaso[\UCsist]  establece la disponibilidad actual con base a la regla de negocio  \BRref{RN31} .
	\UCpaso[\UCsist] despliega \IUref{UI6} solicitando al usuario que seleccione las  condiciones físicas del libro que apliquen.
	\UCpaso[\UCactor]El administrador selecciona las condiciones físicas actuales.
	\UCpaso[\UCactor] El administrador presiona el botón \IUbutton{CONTINUAR}
	\UCpaso[\UCsist]  muestra la comparación entre el último cambio de estado y el actual.
	\UCpaso[\UCsist] obtiene la fecha y hora en la que se hace el registro.
	\UCpaso[\UCsist] solicita a través del \MSGref{MSG6}  que el administrador confirme los datos.\Trayref{A}
	\UCpaso[\UCactor] El administrador confirma los datos presionando el botón  \IUbutton{OK}
	\UCpaso[\UCsist] registra la  disponibilidad del libro y sus condiciones físicas en la 			fecha y hora indicada.\Trayref{B}
	\UCpaso[\UCsist] muestra el mensaje \MSGref{MSG1} como resultado del éxito de registró.\Trayref{C}
	\UCpaso[\UCactor] El administrador confirma presionando el botón  \IUbutton{OK}
			
	\end{UCtrayectoria}		
			
			
		\begin{UCtrayectoriaA}{A}{El administrador presiona el botón \IUbutton{NO} no confirmando los datos.}
			\UCpaso[\UCsist] El sistema nuestra los datos previamente indicados por el administrador.
			\UCpaso[\UCactor] El administrador realiza las modificaciones correspondientes 
			\\ Paso 9trayectoria principal \Trayref{Principal}
		\end{UCtrayectoriaA}
		\begin{UCtrayectoriaA}{B}{No se logró registrar los datos }
			\UCpaso[\UCsist] El sistema verifica el número de intentos con base a la regla del negocio \BRref{RN32}\Trayref{F}
			\UCpaso[\UCsist] El sistema muestra el mensaje \MSGref{E7}.
			Trayectoria principal paso 9\Trayref{Principal}
		\end{UCtrayectoriaA}
		\begin{UCtrayectoriaA}{C}{Los cambios en los estados fiscos del libro ameritan una sanción. }
			\UCpaso[\UCsist] Notifica que el usuario es acreedor de una sanción.
			\UCpaso[\UCsist] El administrador presiona el botón \IUbutton{CONTINUAR}
			\UCpaso[\UCsist] comienza \UCref(CU14)
		\end{UCtrayectoriaA}
	


			
			
			

			
			
			
			
%			\UCpaso[\UCactor] El administrador realiza el proceso de préstamo de libro o devolución   \UCref(CU07) , \UCref(CU09) 
%			\UCpaso[\UCsist] El sistema conserva el identificador del libro e identifica si se trata de una entrega o un préstamo.
%			\UCpaso[\UCsist] El sistema establece la disponibilidad con base a las reglas de negocio \BRref{RN30} \BRref{RN31}{Disponibilidad de un Libro} \Trayref{B}
%			\UCpaso[\UCsist] El sistema despliega \IUref{IU6}{Pantalla Registro} solicitando al usuario que seleccione las  condiciones físicas del libro que aplique.
%			\UCpaso[\UCactor] El administrador presiona el botón \IUbutton{CONTINUAR}
%			\UCpaso[\UCsist] El sistema obtiene la fecha y hora en  que se hace el registro.
%			\UCpaso[\UCsist] El sistema solicita a través del \MSGref{MSG6}  que el administrador confirme los datos.\Trayref{C}
%			\UCpaso[\UCactor] El administrador confirma los datos presionando el botón  \IUbutton{OK}
%			\UCpaso[\UCsist] El sistema registra la  disponibilidad del libro y sus condiciones físicas en la fecha y hora indicada.\Trayref{D}
%			\UCpaso[\UCsist] El sistema muestra el mensaje \MSGref{MSG1} como resultado del éxito de registró.
%			\UCpaso[\UCactor] El administrador confirma presionando el botón  \IUbutton{OK}
%
%	\end{UCtrayectoria}
%			%-------------------------------------- COMIENZA descripción Trayectoria Alternativa.
%		\begin{UCtrayectoriaA}{A}{El administrador desea cambiar el estado de un libro en concreto.}
%			\UCpaso[\UCsist]  El sistema solicita el identificador del libro.
%			\UCpaso[\UCactor] El administrador ingresa el identificador del libro 
%			\UCpaso[\UCsist] El sistema comprueba que exista el libro en los registros.\Trayref{F}
%			\UCpaso[\UCsist] El sistema despliega la \IUref{IU6}{Pantalla Registro} y solicita al administrador que seleccione la disponibilidad del libro.
%			\UCpaso[\UCactor] El administrador indica la disponibilidad del libro.
%			     Trayectoria principal paso 4 \Trayref{Principal}
%		\end{UCtrayectoriaA}
%
%		\begin{UCtrayectoriaA}{B}{El administrador desea modificar la disponibilidad del libro.}
%			\UCpaso[\UCactor] El administrados presiona el botón \IUbutton{MODIFICAR}
%			\UCpaso[\UCsist] El sistema habilita el selector que indica la disponibilidad del libro
%			\UCpaso[\UCactor] El administrador selecciona la disponibilidad correcta.
%			 trayectoria principal paso 4 \Trayref{Principal}
%		\end{UCtrayectoriaA}
%
%		\begin{UCtrayectoriaA}{C}{El administrador presiona el botón \IUbutton{NO} no confirmando los datos.}
%			\UCpaso[\UCsist] El sistema nuestra los datos previamente indicados por el administrador.
%			\UCpaso[\UCactor] El administrador realiza las modificaciones correspondientes 
%			Paso 5 trayectoria principal \Trayref{Principal}
%		\end{UCtrayectoriaA}
%
%		\begin{UCtrayectoriaA}{D}{No se logró registrar los datos }
%			\UCpaso[\UCsist] El sistema verifica el número de intentos con base a la regla del negocio \BRref{RN32}\Trayref{F}
%			\UCpaso[\UCsist] El sistema muestra el mensaje \MSGref{E7}.
%
%			Trayectoria principal paso 9\Trayref{Principal}
%		\end{UCtrayectoriaA}
%	
%		\begin{UCtrayectoriaA}{E}{Excede el Número de Intentos}
%			\UCpaso[\UCsist] El sistema muestra el mensaje \MSGref{MSG7}
%			\UCpaso[\UCactor]El administrador presiona el botón \IUbutton{OK} 
%			Fin del caso de uso 
%		\end{UCtrayectoriaA}
%
%		\begin{UCtrayectoriaA}{F}{No existe registro del libro  }
%			\UCpaso[\UCsist] El sistema muestra el mensaje \MSGref{E4}
%			\UCpaso[\UCactor]El administrador presiona el botón \IUbutton{OK} 
%
%			Trayectoria Alternativa A paso 1  \Trayref{A}
%		\end{UCtrayectoriaA}
				
%-------------------------------------- TERMINA descripción del caso de uso.
